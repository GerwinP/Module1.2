\documentclass[12pt]{article}
\usepackage[dutch]{babel}

\usepackage{amsmath}
\usepackage{graphicx} 
\usepackage{lscape}
\usepackage{marvosym}
\usepackage{hyperref}
\usepackage[T1]{fontenc}
\usepackage{multirow}


\addtolength{\oddsidemargin}{-.5cm}
\addtolength{\evensidemargin}{-.5cm}
\addtolength{\textwidth}{+1cm}
\addtolength{\topmargin}{-1cm}
\addtolength{\textheight}{+1.5cm}

\begin{document}

\title{Programmeer project\\Connect Four}
\author{Gerwin Puttenstein\\s1487779}
\date{\today}
\maketitle
\newpage
\tableofcontents
\newpage
\section{Inleiding}
\section{Inleiding}
Een van de doelen van de afgelopen module, module Software Systemen, was het maken van het project. Als project wordt er een spel gemaakt. Dat spel moet over een server gespeeld kunnen worden en het spel moet zelf controleren of de spelers zich aan de regels houden en of er iemand gewonnen heeft als er een set gedaan is. Het gaat om een bordspel waarbij het niet gaat om geluk maar om strategie zodat er ook een AI voor gemaakt kan worden die door zijn strategie bijna standaard wint.\\
Dit jaar is het spel dat ge\"implementeerd zal worden `vier op een rij'. Dit is een spel voor twee personen. Bij `vier op een rij' is het de bedoeling dat men stenen in een van de zeven rechtopstaande kokers gooit, wanneer een kolom vol is mag hier natuurlijk geen steen meer aan toe gevoegd worden. De spelers doen dit om de beurt en proberen er voor te zorgen dat de andere speler geen vier stenen, horizontaal, verticaal of schuin aansluitend heeft liggen en probeert tegelijkertijd dit wel zelf voor elkaar te krijgen. De eerste die dit lukt heeft gewonnen.\\
Als begin positie is het bord leeg. De persoon die begint is vrij om een rij te kiezen. Ook in elke volgende set is er geen beperking bij het kiezen van de kolom. De rij waarop de steen terecht komt is afhankelijk van het aantal stenen dat er al in die kolom liggen, alle stenen zullen namelijk naar beneden vallen. In het echte spel mag men onderling bepalen wie de eerste set mag doen, in mijn versie van het spel wordt dit door de volgorde waarop de spelers op `play' klikken. Dus de speler die als eerste op 'play' heeft gedrukt mag als eerste beginnen en heeft de rode kleur\\
\newpage 
\section{Bespreking van het totale ontwerp}
\subsection{Klassediagram}

\subsectien{Vereisten}
Er moesten een aantal belangrijke dingen in het programma komen. Deze staan op pagina 18 van de reader en hier heb ik me dan ook zo goed mogelijk aan gehouden.
Voor de server zijn het de volgende vereisten:
\begin{enumerate}
\item Het poortnummer kan je ingeven via de StartServerGUI waar de server daarna naar gaat luisteren, mits het een geldig poortnummer is die nog niet in gebruik is.
\item 
\end{document}