\documentclass[12pt]{article}
\usepackage[dutch]{babel}

\usepackage{amsmath}
\usepackage{graphicx} 
\usepackage{lscape}
\usepackage{marvosym}
\usepackage{hyperref}
\usepackage[T1]{fontenc}
\usepackage{multirow}


\addtolength{\oddsidemargin}{-.5cm}
\addtolength{\evensidemargin}{-.5cm}
\addtolength{\textwidth}{+1cm}
\addtolength{\topmargin}{-1cm}
\addtolength{\textheight}{+1.5cm}

\begin{document}

\title{Programmeer project\\Connect Four}
\author{Gerwin Puttenstein\\s1487779}
\date{\today}
\maketitle
\newpage
\tableofcontents
\newpage
\section{Inleiding}
\section{Inleiding}
Een van de doelen van de afgelopen module, module Software Systemen, was het maken van het project. Als project wordt er een spel gemaakt. Dat spel moet over een server gespeeld kunnen worden en het spel moet zelf controleren of de spelers zich aan de regels houden en of er iemand gewonnen heeft als er een set gedaan is. Het gaat om een bordspel waarbij het niet gaat om geluk maar om strategie zodat er ook een AI voor gemaakt kan worden die door zijn strategie bijna standaard wint.\\
Dit jaar is het spel dat ge\"implementeerd zal worden `vier op een rij'. Dit is een spel voor twee personen. Bij `vier op een rij' is het de bedoeling dat men stenen in een van de zeven rechtopstaande kokers gooit, wanneer een kolom vol is mag hier natuurlijk geen steen meer aan toe gevoegd worden. De spelers doen dit om de beurt en proberen er voor te zorgen dat de andere speler geen vier stenen, horizontaal, verticaal of schuin aansluitend heeft liggen en probeert tegelijkertijd dit wel zelf voor elkaar te krijgen. De eerste die dit lukt heeft gewonnen.\\
Als begin positie is het bord leeg. De persoon die begint is vrij om een rij te kiezen. Ook in elke volgende set is er geen beperking bij het kiezen van de kolom. De rij waarop de steen terecht komt is afhankelijk van het aantal stenen dat er al in die kolom liggen, alle stenen zullen namelijk naar beneden vallen. In het echte spel mag men onderling bepalen wie de eerste set mag doen, in mijn versie van het spel wordt dit door de volgorde waarop de spelers op `play' klikken. Dus de speler die als eerste op 'play' heeft gedrukt mag als eerste beginnen en heeft de rode kleur\\
\newpage 
\section{Bespreking van het totale ontwerp}
\subsection{Klassediagram}

\subsection{Vereisten}
Er moesten een aantal belangrijke dingen in het programma komen. Deze staan op pagina 18 van de reader en hier heb ik me dan ook zo goed mogelijk aan gehouden.
Voor de server zijn het de volgende vereisten:
\begin{enumerate}
\item Het poortnummer kan je ingeven via de StartServerGUI waar de server daarna naar gaat luisteren, mits het een geldig poortnummer is die nog niet in gebruik is.
\item (een goede error wordt gemaakt als port niet kan) Er wordt een error weergeven als er geen geldig poortnummer wordt ingevoerd. Dan terminate het hele programma. 
\item De server ondersteund meerdere games die tegelijkertijd gespeeld kunnen worden. Dit wordt gedaan doordat de Server per twee spelers die willen spelen een nieuwe ServerGame maakt die de Game bij houdt.
\item Er is nu geen TUI die zorgt dat alle berichten naar System.out schrijft, maar een ServerGUI die zorgt dat alle berichten naar een JTextArea worden geschreven die binnenkomen op de server. Ook de berichten die de server stuurt worden naar de JTextArea geschreven.
\item (server protocol)
\end{enumerate}
Voor de client zijn het de volgende eisen die worden gesteld en op de volgende manier zijn ge\"implementeerd.
\begin{enumerate}
 \item Ook hier wordt geen gebruik gemaakt van een TUI, maar van een GUI. Hier is het mogelijk om het adres en de poortnummer van de server waar je op wil spelen kan invoeren en waar je een naam kan invoeren. Ook is het mogelijk om binnen deze GUI te kiezen om standalone te spelen. Dit is een spel tegen de computer.
 \item Er is support voor HumanPlayers en voor ComputerPlayers. Deze maken beide gebruik van een abstracte klasse Player. Hiervoor heb ik een beetje de klassen gevolgd die we hebben gebruikt voor de opdrachen en zelf wat aangepast zodat het beter aansluit op mijn implementatie. De ComputerPlayer bevat een kleine mate van intelligentie, dit houdt in dat de ComputerPlayer een random getal kiest tussen 0 en 6 voor de kolom waar hij een steen wil plaatsen.
 \item (thinking time ComputerPlayer)
 \item (hint)
 \item (als de game klaar is moet de Client een nieuwe game kunnen starten)
 \item 
\end{enumerate}

\subsection{Observer en Model-View-Controller}
Binnen mijn implementatie van het spel is de BoardGUI de observer en de Game klasse de Observable. Wanneer er een zet wordt gedaan op de gui, wordt deze doorgespeeld naar de Game, deze kijkt of dit een geldige zet is en speelt deze zet op het bord, als de zet geldig is. Daarna stuurt de Game een update naar alle Observers. In dit geval is dit alleen de BoardGUI. Als deze een update binnenkrijgt, wordt de zet gedaan op de gui.\\
Ook maak ik gebruikt van het MVC patroon. Binnen mijn implementatie is de Game klasse de model, de BoardGUI klasse de view en de ConnectFourController de controller. De controller maakt een view en een model aan en zorgt ervoor dat updates op de view worden doorgespeeld naar de model en vice versa.

\subsection{Formats voor data opslag en communicatieprotocol}
Een board klasse houdt bij welke plaatsen op het bord zijn bezet met stenen en welke plaatsen nog leeg zijn. Deze velden worden bijgehouden in een array van PlayerColors. Deze array is 42 plaatsen groot, gelijk aan het aantal velden van een vier op een rij spel. Een veld kan leeg zijn, PlayerColor.EMPTY, geel zijn, PlayerColor.YELLOW, of rood zijn, PlayerColor.RED. \\
Voor een online spel houdt een ServerGame bij welke spelers in dat specifieke spel zitten en wie er aan de beurt is. Een ServerGame wordt aangemaakt door de Server als twee spelers willen spelen.\\
Voor het commmunicatieprotocol wordt het protocol gebruikt dat is opgesteld tijdens een projectmeeting. Daar hebben we gekozen om een klasse te gebruiken die commands voor de server, commands voor de client, commands voor beide en errormessages bevat.

\newpage

\section{Discussie per klasse}
Hier volgt een korte toelichting op elke klasse die in mijn implementatie van het spel zitten, opgedeeld per package.\\
\subsection{connectFour}
In deze package zitten alle klassen die nodig zijn voor de kern van het spel.
\subsubsection{Board}
Deze klasse houdt voor elk spel de stand van het bord bij. Ook worden binnen deze klasse de spelregels bijgehouden en zijn er de methoden om te kijken of er een winnaar is. Deze klasse houdt bij of het bord vol is, of het spel voorbij is en wie de winnaar is. \\
De klasse Game gebruik de Boardklasse om het bord aan te kunnen passen, een winnaar te bepalen en te kijken of het spel over is.

\subsubsection{ConnectFourController}
De controller klasse van het spel. Deze klasse maakt een View en een Model aan en zorgt er daarna voor dat er een spel gespeeld kan worden via het MVC patroon. De klasse bevat alleen een contructor om alles aan te maken en een actionPerformed om het klikken op een knop af te handelen.\\
Deze klasse wordt aangeroepen als je standalone wil gaan spelen.

\subsubsection{Game}
De model in het MVC patroon. Deze klasse zorgt ervoor dat er een nieuw spel kan worden gemaakt met twee spelers. Daarnaast houdt hij bij wie er aan de beurt is en voert hij beurten uit voor spelers op een Gui en een bord.\\
De Game wordt aangeroepen door de bovengenoemde controller en door een ServerGame. 

\subsection{gui}
Binnen deze package zijn alle klassen te vinden die zorgen voor het grafische deel van het programma.
\subsubsection{BoardGUI}
De grafische weergave van een bord en tevens de View in het MVC patroon. De BoardGUI bevat in de initi\"ele staat 42 zwarte, niet klikbare knoppen. Deze stellen de velden van een vier op een rij spel voor. Boven elke kolom is een klikbare knop te vinden om een zet te kunnen doen in die kolom.\\
Deze klasse wordt door elke klasse gebruikt die een grafische interface wil voor een bord.


\end{document}